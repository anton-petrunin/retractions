\documentclass[oneside,a4paper, 12pt]{article}
\usepackage{anysize}
\usepackage{retraction}
\usepackage{biber}

\hypersetup{pdftitle={Short retractions of CAT(1) spaces},
pdfauthor={Alexander Lytchak and Anton Petrunin}}

\begin{document}
\title{Short retractions of CAT(1) spaces}
\author{Alexander Lytchak and Anton Petrunin}
\date{}
\maketitle

\begin{abstract}
We construct short retractions of a CAT(1) space to its convex subset.
This construction provides an alternative geometric description of one tool introduced by Wilfrid Kendall. 
We also show how to apply it to Dirichlet problem and its relatives.
\end{abstract}


\section{Introduction}

Recall that a  subset $K$ in a metric space ${\spc{U}}$ is called \emph{weakly convex} if any two points $x,y\in K$ can be connected by a minimizing geodesic in $K$.

Let ${\spc{U}}$ be a metric space and $K\subset {\spc{U}}$.
A distance nonexpanding map $f\:{\spc{U}}\to K$ such that $f(x)=x$ for any $x\in K$ is called a \emph{short retraction to $K$}.
If in addition a local Lipschitz constant of $f$ is strictly less than 1 at any point $p\notin K$, 
then we say that $f$ is a \emph{strictly short retraction from ${\spc{U}}$ to $K$}.  

\begin{thm}{Proposition}\label{thm:retraction:Phi}
Let $\spc{U}$ be a complete length $\CAT(1)$.
 Suppose $K$ is a weakly convex closed subset in $\spc{U}$ and $K\subset \bar B[p,\tfrac\pi2]$ for some point $p\in K$.
Then 
\begin{enumerate}[(a)]
 \item There is a short strong deformation retraction
\[\phi_t\:\bar B[p,\tfrac\pi2]\to K.\]
 \item There is a short retraction 
\[\Phi\:\spc{U}\to K.\]
\end{enumerate}
Moreover, if $\spc{U}$ be a $\CAT(\kappa)$ space for some $\kappa<1$, then both retractions are strictly short.
\end{thm}

This statement is a generalization of the following well known statement about $\CAT(0)$ spaces:
\emph{if ${\spc{U}}$ is a complete $\CAT(0)$ space and $K$ is a closed convex subset in ${\spc{U}}$,
then the closest point projection ${\spc{U}}\to K$ is a short retraction}.


The proof is given in Section~\ref{sec:proofs};
it is done by using a simple trick together with \emph{tractrix flow} which we are about to describe informally.
Suppose that two points $p$ and $q$ in a $\CAT(1)$ space ${\spc{U}}$ are connected to each other by a thread of fixed length $r<\pi$.
Imagine that the point $q$ follow a path and drags $p$ if the thread is tight; 
if the is not tight, then $p$ does not move.
Then the trajectory of the point $p$ will be called $r$-\emph{tractrix} of $p$ with respect to the path
and the family of maps $\phi_t$ that send initial position of $p$ to its position at the time $t$ will be called $r$-\emph{tractrix flow}.

A more formal treatment of tractrix flow is given in Appendix~\ref{sec:Tractrix flow}.
The important properties of the tractrix flow are collected in Proposition~\ref{prop-def}.
In particular, if $r\le \tfrac\pi2$, then $\phi_t$ is short for any $t$ 
and if $r< \tfrac\pi2$ then the local Lipschitz constant of $\phi_t$ at $p$ is strictly less than 1 if $p\ne \phi_t(p)$.

The following statement is proved using the proposition and another trick. 

\begin{thm}{Corollary}\label{cor}
Let $\spc{U}$ be a complete length $\CAT(1)$.
Suppose there is there is a point $p\in \spc{U}$ such that $|p-x|\le \tfrac\pi2$ for any point $x\in \spc{U}$.
Then there is a short retraction 
\[\Psi\:\spc{U}\times \spc{U}\to \Delta,\]
where $\Delta$ is the diagonal in $\spc{U}\times \spc{U}$;
that is, $\Delta=\{\,(x,x)\in \spc{U}\times \spc{U}\,\}$.
 
Moreover, if $\spc{U}$ be a $\CAT(\kappa)$ space for some $\kappa<1$, then $\Psi$ is a strictly short retraction.
\end{thm}

\section{Motivation}

In \cite[(4.1)]{kendall}, Wilfrid Kendall observed that if $\spc{U}$ is a regular geodesic ball or radius $r<\tfrac\pi2$ in a manifold with sectional curvature at most 1, then for appropriate choice of constant $\lambda$, the following function
\[(x,y)\mapsto 
\frac{1+\lambda-\cos|x-y|_{\spc{U}}}{\cos|p-x|_{\spc{U}}\cdot \cos|p-y|_{\spc{U}}}
\]
has convex level sets in ${\spc{U}}\times {\spc{U}}$.
This observation became a useful tool to study Dirichlet problem and its relatives;
it makes possible to extend number of results from Hadamard manifolds to Riemannian manifolds of small size
and more generally to $\CAT(1)$ spaces~\cite{yokota} MORE???.

Our original goal was to make this tool transparent for geometers and \ref{cor} can be considered as a more geometric version of this tool.
In addition, \ref{thm:retraction:Phi} provides a slightly weaker condition for the existence problem than it was known before --- let us describe the difference. 

 
Let us recall the definition of \emph{$1$-complemented metric spaces},
which were introduced by Stephan Wenger \cite{Wenger-1comp}.
A metric space ${\spc{U}}$ is called \emph{$1$-complemented} if for some \emph{non-principal ultrafilter} $\omega$  there exists a short retraction of ultraproduct ${\spc{U}}^{\omega}$ to~${\spc{U}}$.
For the definitions of ultrafilters and ultraproducts, we referring to \cite{Wenger-1comp,Wenger-2comp,akp}. 

Recall that any metric space ${\spc{U}}$ forms a subset in its ultraproduct ${\spc{U}}^{\omega}$, or more precisely, ${\spc{U}}$ admits a canonical isometric embedding into ${\spc{U}}^{\omega}$.
Moreover if ${\spc{U}}$ is $\CAT(\kappa)$ then so is ${\spc{U}}^\omega$. 
Applying these two observations together with the part (\ref{thm:retraction:Phi}) of the theorem, we obtain  the following corollary.

\begin{thm}{Corollary}
Let ${\spc{U}}$ be  a $\CAT(1)$ space.
Assume  there exists some $p\in {\spc{U}}$ such that $|p-x|\le \tfrac\pi2$ for any point $x\in {\spc{U}}$.
Then ${\spc{U}}$ is $1$-complemented.
\end{thm}

Let us list few existence results which hold for $1$-complemented $\CAT(1)$ space ${\spc{U}}$:
\begin{enumerate}[(a)]
\item\label{dirichlet}   The existence of solution $u$ of Dirichlet problem on the minimization of energy 
in $W^{1,2} (\Omega, {\spc{U}})$ on any Lipschitz domain in a Riemannian manifold with prescirbed trace $tr(u)$.
(See \cite{KS,Guo}, 
for the descriptions of the Dirichlet problem.)
\item The existence of a minimal integral $k$-current filling any prescribed boundaryless $(k-1)$ integral current in ${\spc{U}}$. 
(See \cite{Ambrosio,Wenger-1comp} for a discussion of Plateau's problem in  the theory of currents.)
\item   The existence of a conformally parametrized disc $u:D\to {\spc{U}}$ with minimal area, with any  boundary curve $\Gamma$, which is a Jordan curve of finite length in ${\spc{U}}$.
(See \cite{LWplateau, Wenger-2comp} for a discussion classical Plateau's problem for discs in metric spaces.)
\item\label{center} For any Radon measure $\mu$ on ${\spc{U}}$ there exists a center of mass $x\in {\spc{U}}$ for the measure~$\mu$. (See \cite{Sturm, Yokota} for the theory of centers of mass.)
\end{enumerate}

If in the corollary, we would assume strict inequality $|p-x|< \tfrac\pi2$, then the existence results are known  in all the cases (\ref{dirichlet}--\ref{center}).
Moreover the uniqueness hold true under this stronger assumption,
\cite{Yokota,Serbinowski}.
In our boundary case uniqueness definitely fails, since geodesics between points in a round hemisphere are not unique.

The uniqueness in each case can be shown using Corollary \ref{cor}.
Indeed if the solution of one of these problems is not unique, then their product in ${\spc{U}}\times {\spc{U}}$ does not lie in the diagonal.
The latter contradicts existence of the projection $\Psi$ provided by \ref{cor}.

\section{Proofs}\label{sec:proofs}

First let us recall that spherical join $\spc{U}\star\spc{V}$ of two metric spaces $\spc{U}$ and $\spc{V}$
is defined as the unit sphere equipped with the angle metric in the product of Euclidean cones $\Cone \spc{U}\times \Cone\spc{V}$.
If diameters of $\spc{U}$ and $\spc{V}$ do not exceed $\pi$, then $\spc{U}\star\spc{V}$
can be defined as a metric space that admits an onto map $\iota\:\spc{U}\times\spc{V}\times[0,\tfrac\pi2]\to \spc{U}\star\spc{V}$ such that
\begin{align*}
|\iota(u_1,v_1,t_1)&-\iota(u_2,v_2,t_2)|_{\spc{U}\star\spc{V}}=
\\
&=\arccos[\sin t_1\cdot\sin t_2\cdot \cos|u_1-u_2|_{\spc{U}}+\cos t_1\cdot \cos t_2\cdot \cos|v_1-v_2|_{\spc{V}}].
\end{align*}

\parit{Proof of \ref{thm:retraction:Phi}.}
Consider a join of  $K$ with a one-point space, $\spc{J}=K\star \{s\}$.
Since join of two $\CAT(1)$ spaces is $\CAT(1)$, $\spc{J}$ is a $\CAT(1)$ space.
By Reshetnyak's gluing theorem the space $\spc{W}$ glued from ${\spc{U}}$ and $\spc{J}$ along $K$ is a $\CAT(1)$ space;
moreover each space ${\spc{U}}$ and $\spc{J}$ are convex subsets in $\spc{W}$.

Let $\gamma$ be the geodesic path in $\spc{W}$ from $p$ to the pole $s$ of $\spc{J}$.
Note that $\bar B[\gamma(t),\tfrac\pi2]\supset \spc{J}$ for any $t$  
and the family $B_t=\bar B[\gamma(t),\tfrac\pi2]$ is decreasing in $t$.

Note that $B_0=B[p,\tfrac\pi 2]$ and $B_1=\spc{J}$.
According to \ref{prop-def}, the family of $\tfrac\pi2$-tractrix maps $\phi_t$ provide a strong deformation retraction which is short and it is strictly short in case $\spc{U}$ is $\CAT(\kappa)$ for some $\kappa<\tfrac\pi2$.

Since ${\spc{U}}$ is $\CAT(1)$,
given a point $x\in B(p,\pi)$ there is unique geodesic $\gamma_p$ parametrized by its length from $p$ to $x$. 
By $\CAT(1)$ comparison, the map 
\[\Theta(x)=
\begin{cases}
p&\text{if\ }|p-x|_{\spc{U}}\ge \pi,
\\
\gamma_x(\pi-\ell)&\text{if\ }|p-x|_{\spc{U}}< \pi.
\end{cases}
\]
is a short retraction of ${\spc{U}}$ to $\bar B[p,\tfrac\pi2]$, which is strictly short if $\spc{U}$ is $\CAT(\kappa)$ for some $\kappa<\tfrac\pi2$.
Therefore the composition $\phi_1\circ\Theta$ is a short retraction of ${\spc{U}}$ to $K$.
\qeds


\parit{Proof of \ref{cor}.}
Consider the joint $\spc{U}\star\spc{U}$.
It is straightforward to see that 
\[|\iota(u_1,v_1,\tfrac\pi2)-\iota(u_2,v_2,\tfrac\pi2)|^2_{\spc{U}\star\spc{U}}\le |u_1-u_2|^2_{\spc{U}}+|v_1-v_2|_{\spc{U}}^2;\]
that is, the map $(u,v)\mapsto \iota(u,v,\tfrac\pi4)$ induces a short map 
\[\Theta\:{\tfrac1{\sqrt{2}}}\cdot(\spc{U}\times\spc{U})\to\spc{U}\star\spc{U}.\]
(The map $\Theta$ is also length preserving, but we will not need it.)

Note that the diagonal $\tfrac1{\sqrt{2}}\cdot\Delta$ is a convex set in $\tfrac1{\sqrt{2}}\cdot(\spc{U}\times\spc{U})$.
Moreover the restriction of $\Theta$ to $\tfrac1{\sqrt{2}}\cdot\Delta$ is distance preserving;
in particular the image $K=\Theta(\tfrac1{\sqrt{2}}\cdot\Delta)$ is a weakly convex set in $\spc{U}\star\spc{U}$.

Further note that $|p-x|_{\spc{U}\star\spc{U}}\le \tfrac\pi2$ for any $p\in K$.
Applying \ref{thm:retraction:Phi}, we get a short retraction $\Phi\:\spc{U}\star\spc{U}\to \Theta(\tfrac1{\sqrt{2}}\cdot\Delta)$.
Since $\Theta$ is short, it induces the needed short retraction $\Psi\:\spc{U}\times\spc{U}\to \Delta$.
\qeds

\appendix

\section{Tractrix flow}\label{sec:Tractrix flow}

A tractrix can be defined by taking a limit for the following construction.

Fix a positive numbers $r<\pi$ and $\eps<\pi-r$. 
A sequence of points $q_0,\dots,q_n$ in a metric space ${\spc{U}}$ will be called $\eps$-chain if $|q_i-q_{i-1}|_{\spc{U}}<\eps$ for each $i$.
Suppose $p=p_0$ is a point that lies on distance at most $r$ from  $q_0$.
Consider another sequence $p_0,\dots p_n$ defined recursively the following way:
\emph{$p_i$ is the closest point to $p_{i-1}$ that lies on the distance at most $r$ from $q_i$}.
This sequence (if defined) will be called \emph{discrete tractrix} of $p$ with respect to the $\eps$-chain $(q_i)$.

Note that if ${\spc{U}}$ is geodesic, then 
$p_i$ lies on a geodesic from $p_{i-1}$ to $q_i$; 
in this case the sequence $(p_i)$ is defined.
Moreover the triangle inequality implies that $(p_i)$ is an $\eps$-chain.

Note also that $|p_{i-1}-p_i|_{\spc{U}}<\pi$ for each $i$.
Therefore, if ${\spc{U}}$ is $\CAT(1)$, then the geodesic $[p_{i-1},q_i]$ is uniquely defined and 
so is the discrete tractrix $(p_i)$.

Now assume that $\gamma\:[a,b]\to \spc{U}$ is a Lipschitz curve.
Fix a fine partition $a=t_0<\dots<t_n=b$ and set $q_i=\gamma(t_i)$.
Since the partition is fine, we can assume that $(q_i)$ is an $\eps$-chain for any $\eps>0$ fixed in advance.
Moreover, we may assume that each arc $\gamma|_{[t_{i-1},t_i]}$ has diameter at most $\eps$ for each $i$.
Applying $\CAT(1)$ comparison it is straightforward to prove that for finer and finer partition the $p$-chain converges to a curve $\tau\:[a,b]\to \spc{U}$ that will be called $r$-tractrix of $p$ with respect to $\gamma$.
The one-parameter family of maps $\phi_t\:\bar B[\gamma(a),r]\to \bar B[\gamma(t),r]$ defined by $\phi_t\:\tau(a)\mapsto \tau(t)$ will be called $r$-\emph{tractrix flow}.

\begin{thm}{Proposition}\label{prop-def}
The $\tfrac\pi2$-tractrix flow $\phi_t$ is uniquely defined for any Lipschitz curve $\gamma$ in a  $\CAT(1)$ space ${\spc{U}}$.
Moreover,
\begin{enumerate}[(a)]
 \item $\phi_t$ is short for any $t$;
 \item if ${\spc{U}}$ is $\CAT(\kappa)$ for some $\kappa<1$, then there is a positive constant $\eps$ such that the local Lipschitz constant of $\phi_t$ at $p$ is bounded above by $\exp(-\eps\cdot\ell)$, where $\ell=|p-\phi_t(p)|_{\spc{U}}$.
 \item If the family of balls $\bar B[\gamma(t),r]$ is decreasing in $t$ (that is, if $\bar B[\gamma(t_1),r]\supset\bar B[\gamma(t_2),r]$ for $t_1<t_2$), then $\phi_t$ is a strong deformation retraction from $\bar B[\gamma(0),r]$ to $\bar B[\gamma(t),r]$.
\end{enumerate}
\end{thm}

The tractrix is analogous to the pursuer flow introduced and studied by Stephanie Alexander, Richard Bishop, Robert Ghrist and Chanyoung Jun \cite{ABG,jun-thesis,jun}.
More generally, $r$-tractrix flow of 1-Lipschitz curve $\gamma$ could be considered as gradient curves for time-dependent family of function 
\[f_t=- \max\{r,\dist_{\gamma(t)}\}.\]
Such flows were studied in \cite{jun-thesis} only in the $\CAT(0)$ case (despite that $\CAT(\kappa)$ case is equally simple).
Gradient flow for time-independent function were studied in $\CAT(\kappa)$ spaces were studied in \cite{lytchak-open-map} and \cite{ohta-palfia}.
The proof from any of these papers can be easily adapted to prove our proposition. 

{\small\sloppy

\printbibliography[heading=bibintoc]

}

\end{document}
