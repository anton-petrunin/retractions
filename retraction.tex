\documentclass[oneside,a4paper, 12pt]{article}
\usepackage{anysize}
\usepackage{retraction}
\usepackage{biber}

\hypersetup{pdftitle={Short retractions of CAT(1) spaces},
pdfauthor={Alexander Lytchak and Anton Petrunin}}

\begin{document}
\title{Short retractions of CAT(1) spaces}
\author{Alexander Lytchak and Anton Petrunin}
\date{}
\maketitle

\begin{abstract}
We construct short retractions of a CAT(1) space to its small convex subsets.
This construction provides an alternative geometric description of an analytic tool introduced by Wilfrid Kendall.

Our construction use \emph{tractrix flow} which cab be defined as a gradient flow for a family of functions of certain type.
In the appendix we develop gradient flow for families of semiconcave functions which should be also useful elsewhere.
\end{abstract}


\section{Introduction}


Recall that a  subset $K$ in a metric space ${\spc{U}}$ is called \emph{weakly convex} if any two points $x,y\in K$ can be connected by a minimizing geodesic in $K$.

Let ${\spc{U}}$ be a metric space and $K\subset {\spc{U}}$.
A distance nonexpanding map $f\:{\spc{U}}\to K$ such that $f(x)=x$ for any $x\in K$ is called a \emph{short retraction to $K$}.
If in addition a local Lipschitz constant of $f$ is strictly less than 1 at any point $x\notin K$, 
then we say that $f$ is a \emph{strictly short retraction from ${\spc{U}}$ to $K$}.

\begin{thm}{Proposition}\label{thm:retraction:Phi}
Let $\spc{U}$ be a complete length $\CAT(\kappa)$ space.
Suppose $K$ is a weakly convex closed subset in $\spc{U}$ and there is $p$ such that $|p-x|\le \tfrac\pi2$ for any point $x\in K$.


\begin{enumerate}[(a)]
 \item If $p\in K$ and $\kappa\le 1$, then there is a short retraction 
$\Phi\:\spc{U}\to K$.
\item If $\kappa<1$, then there is a strictly short retraction 
$\Phi\:\spc{U}\to K$.
\end{enumerate}
\end{thm}

This statement is a generalization of the following well known statement about $\CAT(0)$ spaces:
\emph{If ${\spc{U}}$ is a complete length $\CAT(0)$ space and $K$ is a closed convex subset in ${\spc{U}}$,
then the closest point projection ${\spc{U}}\to K$ is a short retraction.
Moreover, if $\spc{U}$ is a $\CAT(\kappa)$ space for some $\kappa<0$, then  the closest point projection is a strictly short retraction}.

\begin{thm}{Corollary}\label{cor}
Let $\spc{U}$ be a complete length $\CAT(\kappa)$ space.
Denote by $\Delta$ the diagonal in $\spc{U}\times \spc{U}$;
that is, $\Delta=\{\,(x,x)\in \spc{U}\times \spc{U}\,\}$.

Suppose there is a point $p\in \spc{U}$ such that $|p-x|\le \tfrac\pi2$ for any point $x\in \spc{U}$.
\begin{enumerate}[(a)]
\item
If $\kappa\le 1$, then there is a short retraction $\Psi\:\spc{U}\times \spc{U}\to \Delta$.
\item If $\kappa<1$, then there is a strictly short retraction $\Psi\:\spc{U}\times \spc{U}\to \Delta$.
\end{enumerate}

\end{thm}

It is well known that if $\spc{U}$ be a complete length $\CAT(0)$ space, then the midpoint map $\spc{U}\times \spc{U}\to \spc{U}$ is $\tfrac1{\sqrt{2}}$-Lipschitz and therefore it induces a short retraction $\spc{U}\times \spc{U}\to\Delta$. 
The corollary provides an analogous statement for $\CAT(1)$ spaces.


\parbf{Motivation.}
In \cite[(4.1)]{kendall}, Wilfrid Kendall observed that if $\spc{B}$ is a regular geodesic ball or radius $r<\tfrac\pi2$ in a manifold with sectional curvature at most 1, then for appropriate choice of constant $\lambda$, the following function
\[(x,y)\mapsto 
\frac{1+\lambda-\cos|x-y|_{\spc{B}}}{\cos|p-x|_{\spc{B}}\cdot \cos|p-y|_{\spc{B}}}
\]
has convex level sets in ${\spc{B}}\times {\spc{B}}$.
He also shows the existence of a nonnegative convex function on ${\spc{B}}\times {\spc{B}}$ that vanish only on the diagonal \cite[(4.2)]{kendall}.
These observations became a useful tool to study Dirichlet problem and its relatives;
it makes possible to extend number of results from Hadamard manifolds to Riemannian manifolds of small size
and more generally to $\CAT(1)$ spaces~\cite{yokota,BFHMSZ,fuglede,serbinowski}. 

Our original goal was to make this tool transparent for geometers.
Corollary~\ref{cor} can be considered as a more geometric version of this tool.
While Kendall's condition is optimal for uniqueness and regularity questions, the existence statements can be derived from  Proposition~\ref{thm:retraction:Phi} in a
slightly greater generality, as  we are going to explain now.

The following definition was introduced by Stefan Wenger \cite{Wenger-1comp};
for the definitions of ultrafilters and ultracompletions we refer to \cite{Wenger-1comp,guo-wenger,akp}.

A metric space ${\spc{U}}$ is called \emph{$1$-complemented} if for some \emph{non-principal ultrafilter} $\omega$  there exists a short retraction of ultracompletion ${\spc{U}}^{\omega}$ to~${\spc{U}}$.
The statement makes sense since any metric space ${\spc{U}}$ forms a subset in its ultracompletion ${\spc{U}}^{\omega}$, or more precisely, ${\spc{U}}$ admits a canonical isometric embedding into ${\spc{U}}^{\omega}$. 

Recall that if ${\spc{U}}$ is $\CAT(\kappa)$, then so is~${\spc{U}}^\omega$. 
Applying these observations together with Proposition~\ref{thm:retraction:Phi}, we obtain

\begin{thm}{Theorem}\label{thm:complemented}
Let ${\spc{U}}$ be a complete length $\CAT(1)$ space.
Assume  there exists some $p\in {\spc{U}}$ such that $|p-x|\le \tfrac\pi2$ for any point $x\in {\spc{U}}$.
Then ${\spc{U}}$ is $1$-complemented.
\end{thm}

Let us list few existence results which follow from the theorem, assuming that the space ${\spc{U}}$ as above:
\begin{enumerate}[(a)]
\item\label{dirichlet}   The existence of solution $u$ of Dirichlet problem on the minimization of energy 
in $W^{1,2} (\Omega, {\spc{U}})$ on any Lipschitz domain $\Omega$ in a Riemannian manifold with prescirbed trace $tr(u)$.
(See \cite{KS} and \cite[Theorem 1.4]{guo-wenger}).
\item The existence of a minimal integral $k$-current filling of any prescribed boundaryless $(k-1)$ integral current in ${\spc{U}}$. 
(See \cite{Ambrosio} and \cite[Theorem 3.3]{Wenger-1comp}.)
\item   The existence of a conformally parametrized disc $u:D\to {\spc{U}}$ of minimal area for a given boundary curve $\gamma$, which is a Jordan curve of finite length in ${\spc{U}}$.
(See \cite{LWplateau} and \cite[Theorem 1.2]{guo-wenger}).
\item\label{center} For any Radon measure $\mu$ on ${\spc{U}}$ there exists a center of mass $x\in {\spc{U}}$ for the measure~$\mu$ \cite{Sturm, yokota}.
\end{enumerate}

If in the theorem, we assume strict inequality $|p-x|< \tfrac\pi2$, then the existence results are known  in all the cases (\ref{dirichlet}--\ref{center}).
Moreover the uniqueness hold true under this stronger assumption
\cite{yokota,serbinowski}.
In our boundary case uniqueness definitely fails, since geodesics between points in a round hemisphere are not unique.

The uniqueness in each case can be shown using Corollary \ref{cor}.
Indeed if there are different solutions of one of these problems, then their product in ${\spc{U}}\times {\spc{U}}$ does not lie in the diagonal.
The latter contradicts the existence of the strictly short retraction $\Psi$ provided by Corollary \ref{cor}.

\parbf{About the proofs.}
We use a new tool which we call $r$-\emph{tractrix flow}; it gives a family of maps $\phi_t$ for a given rectifiable curve $t\mapsto\gamma(t)$.
The important properties of the tractrix flow are collected in Proposition~\ref{prop-def}.
In particular, (1) if $\spc{U}$ is $\CAT(1)$ and $r\le \tfrac\pi2$, then $\phi_t$ is short for any $t$, 
and (2) if $r< \tfrac\pi2$, then the local Lipschitz constant of $\phi_t$ at $p$ is strictly less than 1 if $p\ne \phi_t(p)$.

In the proof of Proposition~\ref{thm:retraction:Phi}, the tractrix flow is applied in a space arising by gluing $\spc{U}$ with a spherical cone over $K$ along $K$;
this space is $\CAT(1)$ by Reshetnyak's gluing theorem.

In the proof of Corollary~\ref{cor} the additional trick consists in identifying the product space $\spc{U}\z\times \spc{U}$ with a subset of the spherical join $\spc{U}\star \spc{U}$ and applying  Proposition~\ref{thm:retraction:Phi} to the latter.

The tricks in both proofs show that it is useful to consider singular spaces even in the case when the original space ($\spc{U}$) is smooth;
this is a powerful freedom of Alexandrov's world.
More involved examples of such arguments are given by Dmitry Burago, Sergei Ferleger, and Alexey Kanonenko \cite{BFK}, and Stephan Stadler~\cite{stadler}.

\section{Tractrix flow}\label{sec:Tractrix flow}

For $\CAT(\kappa)$ spaces, we will follow the conventions in \cite{akp}.

First let us describe the tractrix flow informally.
Suppose that two points $p$ and $q$ in ${\spc{U}}$ are connected to each other by a thread of fixed length $r$.
Imagine that the point $q$ follow the curve $\gamma$ and drags $p$ if the thread is tight; 
if the thread is not tight, then $p$ does not move.
Then the trajectory of the point $p$ will be called $r$-\emph{tractrix of $p$ with respect to~$\gamma$}.
The family of maps $\phi_t$ that send the initial position of $p$ to its position at the time $t$ will be called $r$-\emph{tractrix flow}.

More formally, suppose $\gamma\:[a,b]\to \spc{U}$ is a 1-Lipschitz curve. 
An $r$-tractrix with respect to $\gamma$ is defined as a gradient curve for the time-dependent family of functions 
\[f_t=- \max\{r,\dist_{\gamma(t)}\}\]
if the initial point lies in $\bar B[\gamma(a),r]$;
here $\bar B[x,r]$ denotes the closed ball of radius $r$ centered at $x$ and $\dist_{x}$ denotes the distance function from the point $x$.

The $r$-\emph{tractrix flow} with respect to $\gamma$ is defined as a family of maps
\[\phi_t\:\bar B[\gamma(a),r]\z\to\bar B[\gamma(t),r]\]
defined by $\phi_t\:p\mapsto \tau(t)$
for an $r$-tractrix $\tau\:[a,b]\to \spc{U}$ that starts at $p$; that is, $\tau(a)=p$.

The following proposition includes the properties of the tractrix flow which will be used further in the paper.

\begin{thm}{Proposition}\label{prop-def}
Let $\spc{U}$ be a complete length $\CAT(\kappa)$ space for $\kappa\le 1$ and $r<\pi$.
Then the $r$-tractrix flow $\phi_t\:\bar B[\gamma(a),r]\to\bar B[\gamma(t),r]$ is uniquely defined for any  1-Lipschitz curve $\gamma\:[a,b]\to \spc{U}$.

Moreover, if $r=\tfrac\pi2$, then
\begin{enumerate}[(a)]
 \item\label{non-strict} $\phi_t$ is short for any $t$;
 \item\label{strict} if $\kappa<1$, then there is a positive constant $\eps$ such that the local Lipschitz constant of $\phi_t$ at $p$ is bounded above by $\exp(-\eps\cdot\ell)$, where $\ell=|p-\phi_t(p)|_{\spc{U}}$.
 \item\label{sharafutdinov} If the family of balls $B_t=\bar B[\gamma(t),r]$ is decreasing in $t$ (that is, if $B_{t_1}\supset B_{t_2}$ for $t_1<t_2$), then $\phi_t$ is a strong deformation retraction from $B_a$ to $B_b$.
\end{enumerate}
\end{thm}

Historically the first relative of tractrix flow
is the so called \emph{Sharafutdinov's retraction} \cite{sharafutdinov} --- a family of maps associated to a continuous family of convex sets (in our case these sets are the balls $B_t$). 
Second relative is the \emph{pursuer flow} introduced and studied by Stephanie Alexander, Richard Bishop, Robert Ghrist and Chanyoung Jun \cite{ABG,jun-thesis,jun,jun:grad}.

Let us also mention that time-dependent gradient flows were studied by Chanyoung Jun \cite{jun-thesis,jun:grad} and  by Lucas C. F. Ferreira and Julio C. Valencia-Guevara \cite{ferreira-valencia}.

Unfortunately Proposition~\ref{prop-def} does not follow directly from these papers
and we had to develop time-dependent gradient flow our self.
The proposition follows from the distance estimate (\ref{Distance estimate}) and the existence of gradient curves (\ref{prop:time-dependent}).
We also give the following sketch of a very direct proof;
a reader experienced with $\CAT(\kappa)$ comparison should be able to fill the details.

\parit{Stretch of proof.}
First let us define a \emph{discrete} tractrix.
Fix positive numbers $r<\pi$ and $\eps<\pi-r$. 
A sequence of points $q_0,\dots,q_n$ in  ${\spc{U}}$ will be called an $\eps$-chain if $|q_i-q_{i-1}|_{\spc{U}}<\eps$ for each $i$.
Suppose $p=p_0$ is a point that lies on distance at most $r$ from  $q_0$.
Consider a sequence $p_0,\dots,p_n$ defined recursively the following way:
\emph{$p_i$ is the closest point to $p_{i-1}$ on the geodesic $[p_{i-1}q_i]$ that lies on the distance at most $r$ from $q_i$}.
The sequence $p_0,\dots,p_n$ will be called \emph{discrete tractrix} of $p$ with respect to the $\eps$-chain $q_0,\dots,q_n$.

Note that the triangle inequality implies that $(p_i)$ is an $\eps$-chain.
Since $|p_{i-1}-q_i|_{\spc{U}}<\pi$ for each $i$, the geodesic $[p_{i-1},q_i]$ is uniquely defined and 
so is the discrete tractrix $(p_i)$.

Now assume that $\gamma\:[a,b]\to \spc{U}$ is a Lipschitz curve.
Fix a fine partition $a=t_0<\dots<t_n=b$ and set $q_i=\gamma(t_i)$.
Since the partition is fine, we can assume that $(q_i)$ is an $\eps$-chain for any $\eps>0$ fixed in advance.
Moreover, we may assume that each arc $\gamma|_{[t_{i-1},t_i]}$ has diameter at most $\eps$ for each $i$.
Applying $\CAT(1)$ comparison it is straightforward to prove that for finer and finer partition the $p$-chain converges to a curve $\tau\:[a,b]\to \spc{U}$ which is an $r$-tractrix of $p$ with respect to $\gamma$.
The needed distance estimates follow from the corresponding estimates in the discrete case.
\qeds







\section{Proofs}\label{sec:proofs}

Recall that spherical join $\spc{U}\star\spc{V}$ of two metric spaces $\spc{U}$ and $\spc{V}$
is defined as the unit sphere equipped with the angle metric in the product of Euclidean cones $\Cone \spc{U}\times \Cone\spc{V}$. 
If diameters of $\spc{U}$ and $\spc{V}$ do not exceed $\pi$, then $\spc{U}\star\spc{V}$
can be defined as a metric space that admits an onto map $\iota\:\spc{U}\times\spc{V}\times[0,\tfrac\pi2]\to \spc{U}\star\spc{V}$ such that
\[
\begin{aligned}
|\iota(u_1,v_1,t_1)&-\iota(u_2,v_2,t_2)|_{\spc{U}\star\spc{V}}=
\\
&=\arccos[\sin t_1\cdot\sin t_2\cdot \cos|u_1-u_2|_{\spc{U}}+\cos t_1\cdot \cos t_2\cdot \cos|v_1-v_2|_{\spc{V}}].
\end{aligned}
\eqlbl{join-formula}
\]

Recall that join of two $\CAT(1)$ spaces is $\CAT(1)$ \cite[Corollary 3.14]{bridson-haefliger}.

\parit{Proof of \ref{thm:retraction:Phi}.}
Consider a join of  $K$ with a one-point space, $\spc{J}=K\star \{s\}$.
Since $\spc{J}$ is a $\CAT(1)$ space,
by Reshetnyak's gluing theorem \cite[8.9.1]{akp}, the space $\spc{W}$ glued from ${\spc{U}}$ and $\spc{J}$ along $K$ is a $\CAT(1)$ space;
moreover each space ${\spc{U}}$ and $\spc{J}$ is a convex subsets in $\spc{W}$.

\begin{wrapfigure}{o}{50 mm}
\vskip-0mm
\centering
\includegraphics{mppics/pic-2}
\end{wrapfigure} 


Let $\gamma$ be the geodesic in $\spc{W}$ from $p$ to the pole $s$ of $\spc{J}$.
Set $B_t=\bar B[\gamma(t),\tfrac\pi2]_{\spc{W}}$, then
\begin{itemize}
\item $B_0=\bar B[p,\tfrac\pi2]_{\spc{U}}\cup \spc{J}$,
\item $B_{\frac\pi2}=\spc{J}$, in particular $B_{\frac\pi2}\cap \spc{U}=K$,
\item the family $B_t$ is decreasing in $t$.
\end{itemize}
According to \ref{prop-def}(\ref{sharafutdinov}), the $\tfrac\pi2$-tractrix flow $\phi_t$ is a strong deformation retraction of $B_0$ to $B_{\frac\pi2}$.
By \ref{prop-def}(\ref{non-strict}) $\phi_{\frac\pi2}$ is a short.
If $\kappa<1$, then by \ref{prop-def}(\ref{strict}), $\phi_{\frac\pi2}$ is a strictly short retraction.

Since ${\spc{U}}$ is $\CAT(1)$,
given a point $x\in B(p,\pi)$ there is unique geodesic $\gamma_x$ parametrized by its length from $p$ to $x$. 
By $\CAT(1)$ comparison, the map 
\[\Theta(x)=
\begin{cases}
p&\text{if\ }|p-x|_{\spc{U}}\ge \pi,
\\
\gamma_x(\pi-|p-x|_{\spc{U}})&\text{if\ }|p-x|_{\spc{U}}< \pi.
\end{cases}
\]
is a short retraction of ${\spc{U}}$ to $\bar B[p,\tfrac\pi2]_{\spc{U}}=B_0\cap \spc{U}$.
Moreover $\Theta$ is strictly short retraction if $\kappa<1$.

Therefore the composition $\Phi=\phi_{\frac\pi2}\circ\Theta$ induces a short retraction of ${\spc{U}}$ to $K$
which is strictly short if $\kappa<1$.

Finally, we need to take care of the case $\kappa<1$ and $p\notin K$.
Denote by $\bar p\in K$ the closest point to $p$; by $\CAT(\kappa)$ comparison it is uniquely defined.
Note that $|\bar p-x|_{\spc{U}}< |p-x|_{\spc{U}}$ for any $x\in K$;
therefore $K\subset B(\bar p,\tfrac\pi2)$. 
It remains to apply the construction above with $\bar p$ instead $p$.
\qeds


\parit{Proof of \ref{cor}.}
Consider the spherical join $\spc{U}\star\spc{U}$ and the map $\iota$ described at the beginning of the section. 
Note that \ref{join-formula} implies that the map $(u,v)\mapsto \iota(u,v,\tfrac\pi4)$
induces a length preserving map 
\[\Theta\:{\tfrac1{\sqrt{2}}}\cdot(\spc{U}\times\spc{U})\to\spc{U}\star\spc{U}.\]
In particular, $\Theta$ is short.

Note that the diagonal $\tfrac1{\sqrt{2}}\cdot\Delta$ is a convex set in $\tfrac1{\sqrt{2}}\cdot(\spc{U}\times\spc{U})$.
Moreover \ref{join-formula} implies that the restriction of $\Theta$ to $\tfrac1{\sqrt{2}}\cdot\Delta$ is distance preserving.
In particular the image $K=\Theta(\tfrac1{\sqrt{2}}\cdot\Delta)$ is a weakly convex set in $\spc{U}\star\spc{U}$.

Further note that $|p-x|_{\spc{U}\star\spc{U}}\le \tfrac\pi2$ for any $p\in K$.
Applying \ref{thm:retraction:Phi}, we get a short retraction $\Phi\:\spc{U}\star\spc{U}\to K$.
Since $\Theta$ is short, it induces the needed short retraction $\Psi\:\spc{U}\times\spc{U}\to \Delta$.

Finally, by \ref{thm:retraction:Phi}, if $\kappa<1$, then $\Phi$ is a strictly short retraction and therefore so is $\Psi$.
\qeds

\appendix
\section{Time-dependent gradient flow}

\parbf{Time-independent flow.}
Suppose $\spc{U}$ is a complete length $\CAT(\kappa)$ space.

For a locally Lipschitz semiconcave function $f$ defined on an open set $\Dom f\subset \spc{U}$, the differential $d_pf\:\T_p\to\RR$ is defined at each point $p\in \Dom f$;
it is a concave, Lipschitz, and positive-homogeneous of degree 1 function on the tangent space at $p$. %ADDREF

Further, for any point $p\in \Dom f$ there is unique tangent vector $u\in \T_p$
such that the following two conditions
\[
\begin{aligned}
\langle u,w\rangle &\ge \nabla_{p}f(w),
\\
\langle u,u\rangle &= \nabla_{p}f(u).
\end{aligned}
\eqlbl{<u,w>}
\]
hold for any tangent vector $w\in \T_p$.%
\footnote{Here \emph{tangent vector} means \emph{element of tangent cone}, we use this term despite the tangent cone is not a vector space.
The scalar product $\langle u,w\rangle$ is defined as $|u|\cdot|w|\cdot\cos\theta$ where $\theta$ is the angle between the vectors.}
The vector $u$ is called \emph{gradient of $f$ at $p$}; briefly $u=\nabla_pf$.
%ADDREF

A locally Lipschitz map $t\mapsto \alpha(t)$ is called \emph{$f$-gradient curve} if it satisfies the following equation
\[\alpha^+(t)=\nabla_{\alpha(t)}f\eqlbl{grad-flow}\]
for any $t$, where $\alpha^+(t)\in \T_{\alpha(t)}$ denotes the \emph{right velocity vector}; that is,
\[\alpha^+(t)=\lim_{\eps\to 0+} \frac{\log_{\alpha(t)}[\alpha(t+\eps)]}\eps,\]
where $v=\log_pq$ if the vector $v\in\T_p$ points form $p$ in the direction of $q$ and $|v|=|p-q|$.

The following proposition can be extracted from \cite[Theorem 1.7]{lytchak-open-map}.

\begin{thm}{Proposition}\label{prop:time-independent}
Let ${\spc{U}}$ be a complete length $\CAT(\kappa)$ space and
$f$ a locally Lipschitz and semiconcave function defined on an open set $\Dom f\subset {\spc{U}}$.
Then for any point $p\in \Dom f$  
there is unique $f$-gradient curve $t\mapsto\alpha(t)$ defined on a maximal semiopen interval $[0,\eps)$ for some $\eps>0$ such that $\alpha(0)=p$. 
\end{thm}

Note that the proposition can be applied recursively to obtain a gradient curve defined on a maximal semiopen interval.

\parbf{Time-dependent gradient flow.}
Our next aim is to define gradient flow for a time-dependent family of functions and prove an analog of Proposition~\ref{prop:time-independent} for this flow.
Let $f_t$ be family of functions defined on open subsets $\Dom f_t$ of~$\spc{U}$.
More precisely, we assume that the parameter $t$ lies in a real interval $\II$ and 
\[\Omega=\set{(x,t)\in\spc{U}\times \II}{x\in\Dom f_t}\]
is an open subset in $\spc{U}\times \II$.

The family $f_t$ is called continuous if for any closed set $K$ and closed interval $[a,b]$ such that $K\times[a,b]\subset \Omega$ the family of the restrictions $f_t|_K$ is uniformly continuous in $[a,b]$.
A family of functions $f_t$ is called \emph{Lipschitz} if 
the function $x\mapsto f_t(x)$ is $L$-Lipschitz for some fixed $L$.
It is called \emph{locally Lipschitz} if each $(p_0,t_0)\in \Omega$ there is a neighborhood $\Omega'$ such that 
the restriction of the family to $\Omega'$ is Lipschitz.

A family of functions $f_t$ will be called \emph{$\lambda$-concave} if 
the function $x\mapsto f_t(x)$ is $\lambda$-concave for each $t$.
A family $f_t$ is called \emph{semiconcave} if for for each $(p_0,t_0)\in \Omega$ there is a neighborhood $\Omega'$ and $\lambda\in\RR$ such that the restriction of $f_t$ to $\Omega'$ is $\lambda$-concave. 

Note that one can not expect that a direct generalization of equation \ref{grad-flow} holds for a family of functions $f_t$.

For example, consider a $1$-Lipschitz curve $\alpha$ in the real line. 
Note that it is reasonable to assume that $\alpha$ is an $f_t$-gradient curve for the family $f_t(x)=-|x-\alpha(t)|$.
(Indeed $\alpha$ can be realized as a limit of the gradient curves for a family of functions obtained by smoothing $f_t$.)
On the other hand, $\alpha^+(t)$ might be undefined,
but even if it is defined, $\alpha^+(t)\ne0$ in general, while $\nabla_{\alpha(t)} f_t\equiv0$.


Instead we define $f_t$-gradient curve as a curve $\alpha$ that satisfies the following inequality 
for $\eps>0$ and any point $p\in\spc{U}$ 
\[\dist_p\circ\alpha(t+\eps)\le \dist_p\circ\alpha(t)-\eps\cdot d_{\alpha(t)}f_t(\dir{\alpha(t)}p)+o(\eps),\eqlbl{def:tdflow}\]
where $\dir qp\in \T_{q}$ denotes a unit tangent vector at $q$ in the direction of $p$.

Note that if $\alpha^+(t)=\nabla_{\alpha(t)}f_t$, then \ref{def:tdflow} holds;
this follow from \ref{<u,w>}.
On the other hand, the example above shows that the converse does not hold;
that is, \ref{def:tdflow} generalize the definition~\ref{grad-flow}.

\begin{thm}{Distance estimate}\label{Distance estimate}
Let $f_t$ and $h_t$ be two families of $\lambda$-concave functions on a $\CAT(\kappa)$ space $\spc{U}$ and $s\ge 0$.
Assume $f_t$ and $h_t$ have common domain $\Omega\subset {\spc{U}}\times \RR$ and $|f_t(x)-h_t(x)|\le s$ for any $(x,t)\in \Omega$.
Assume $t\mapsto \alpha(t)$ and $t\mapsto \beta(t)$ are $f_t$- and $h_t$-gradient curves correspondingly, set $\ell(t)=|\alpha(t)-\beta(t)|_{\spc{U}}$.
If for some $t$ a minimizing geodesic $[\alpha(t),\beta(t)]$ lies in $\set{x\in {\spc{U}}}{(x,t)\in \Omega}$, then
\[\ell'(t)\le \lambda\cdot\ell(t)+2\cdot s/\ell(t),\]
assuming that the left hand side is defined.

In particular the inequality holds for any $t\in\II$ if $\Omega\supset B(p,2\cdot r)\times \II$ and $\alpha(t),\beta(t)\z\in B(p, r)$ for any $t\in \II$.
\end{thm}

Note that $t\mapsto \ell$ is a Lipschitz function.
Therefore by Rademacher's theorem, its derivative $\ell'$ is defined almost everywhere and it satisfies the fundamental theorem of calculus.
In particular, it implies uniqueness of the future of gradient curve with given initial point.
It also makes possible to estimate the distance between two gradient curves for close functions which implies convergence of gradient curves $f_t^n$-gradient curves if a sequence of $L$-Lipschitz and $\lambda$-concave families $f^n_t$ converges uniformly to a family $f_t$ as $n\to \infty$. 

\parit{Proof.}
Choose a time moment $t$.
Let $p$ be the midpoint of the geodesic $[\alpha(t)\beta(t)]$.
Let $\gamma\:[0,\ell(t)]\to \spc{U}$ be an arc length parameterization of $[\alpha(t)\beta(t)]$ from $\alpha(t)$ to $\beta(t)$.
Note that $d_{\alpha(t)}f(\dir{\alpha(t)}{p})$ is the right derivative of $f\circ\gamma$ at $0$
and $-d_{\alpha(t)}g(\dir{\beta(t)}p)$ is the left derivative of $g\circ\gamma$ at $\ell(t)$.
Since $f$ and $g$ are $\lambda$-concave,
\begin{align*}
f(\beta(t))&\le f(\alpha(t))+\ell(t)\cdot d_{\alpha(t)}f(\dir{\alpha(t)}{p}) +\tfrac12\cdot\lambda\cdot\ell(t)^2,
\\
g(\alpha(t))&\le g(\beta(t))+\ell(t)\cdot d_{\alpha(t)}g(\dir{\beta(t)}p) +\tfrac12\cdot\lambda\cdot\ell(t)^2,
\end{align*}
Adding these inequalities up and taking into account that $|f(x)-g(x)|<s$ for any $x$, we conclude that 
\[d_{\alpha(t)}f(\dir{\alpha(t)}{p})+d_{\alpha(t)}g(\dir{\beta(t)}p)\ge \lambda\cdot \ell(t)+2\cdot s/\ell(t).\]

Applying the triangle inequality and the definition of gradient curve at $p$, we get that
\begin{align*}
\ell(t+\eps)&=|\alpha(t+\eps)-\beta(t+\eps)|\le
\\
&\le |\alpha(t+\eps)-p|+|\beta(t+\eps)-p|\le 
\\
&\le |\alpha(t)-p|-\eps\cdot d_{\alpha_t}f(\dir{\alpha(t)}{p})+|\beta(t+\eps)-p|-\eps\cdot d_{\beta_t}g(\dir{\beta(t)}p)+o(\eps)=
\\
&=\ell(t)-\eps\cdot(\lambda\cdot \ell(t)+2\cdot s/\ell(t))+o(\eps)
\end{align*}
for $\eps>0$; whence the statement.
\qeds

\begin{thm}{Proposition}\label{prop:def-time-dependent}
Let $\spc{U}$ be a complete length $\CAT(\kappa)$ space.
Suppose $f_t$ is a family of $\lambda$-concave functions for $t\in [a,b)$ and $\Dom f_t\supset B(z,2\cdot r)$ for some fixed $z\in\spc{U}$, $r>0$ and any~$t$.

Let $\alpha\:[a,b)\to B(z,r)$ be a Lipschitz curve.
Then $\alpha$ is a $f_t$-gradient if and only if 
\[\dist_p\circ\alpha(t+\eps)\le \dist_p\circ\alpha(t)-\eps\cdot \left[\tfrac{f(p)-f\circ\alpha(t)}{|p-\alpha(t)|}-\tfrac\lambda2\cdot |p-\alpha(t)|\right]+o(\eps)\eqlbl{def:tdflow-plus}\]
for any $p\in B(z,r)$.
\end{thm}

\parit{Proof.}
Note that geodesics $[\alpha(t)p]$ lie in $\Dom f_t$ for any $t$.

Since $f_t$ is $\lambda$ concave, we have that 
\[d_{\alpha(t)}f_t(\dir{\alpha(t)}p)\ge \tfrac{f(p)-f\circ\alpha(t)}{|p-\alpha(t)|}-\tfrac\lambda2\cdot |p-\alpha(t)|.\]
Whence the only-if part follows.

Given a point $p\in \spc{U}$ and $t$,
consider a point $\bar p\in [\alpha(t)p]$.
Applying \ref{def:tdflow-plus} for $\bar p$ and the triangle inequality, we get
\[\dist_p\circ\alpha(t+\eps)\le \dist_p\circ\alpha(t)-\eps\cdot \left[\tfrac{f(\bar p)-f\circ\alpha(t)}{|\bar p-\alpha(t)|}-\tfrac\lambda2\cdot |\bar p-\alpha(t)|\right]+o(\eps).\]
By taking $\bar p$ close to $\alpha(t)$,
the value $\tfrac{f(\bar p)-f\circ\alpha(t)}{|\bar p-\alpha(t)|}-\tfrac\lambda2\cdot |\bar p-\alpha(t)|$ can be made arbitrary close to $d_{\alpha(t)}f_t(\dir{\alpha(t)}p)$.
Therefore, given $\delta>0$, the following inequality
\[\dist_p\circ\alpha(t+\eps)\le \dist_p\circ\alpha(t)-\eps\cdot d_{\alpha(t)}f_t(\dir{\alpha(t)}p)+\eps\cdot\delta\]
holds for all sufficiently small positive values $\eps$.
Therefore \ref{def:tdflow} holds.
\qeds


Now we are ready to formulate and prove an analog of Proposition~\ref{prop:time-independent} for time-dependent family.

\begin{thm}{Proposition}\label{prop:time-dependent}
Let $\spc{U}$ be a complete length $\CAT(\kappa)$ space and
$\{f_t\}_{t\in[a,b)}$ a continuous locally Lipschitz and semiconcave family of function defined on an open sets $\Dom f_t\subset {\spc{U}}$.
Then for any initial point $p\in \Dom f_a$ and sufficiently small $\eps=\eps(p)>0$ there is a unique $f$-gradient curve $t\mapsto\alpha(t)$ defined on a semiopen interval $[a,a+\eps)$. 
\end{thm}

As well as its time-independent version, this proposition can be applied recursively to obtain a unique gradient curve defined on a maximal semiopen interval.

\parit{Proof.}
By \ref{prop:time-independent}, the proposition holds if $f_t$ is a constant family.
Moreover it holds if $f_t$ is piecewise constant;
that is, if there is a partition of interval $[a,b)$ into a finite collection of semiopen subintervals such that $f_t$ is time-independent on each of subintervals and its domain is fixed.

Fix $\eps>0$ sufficiently small so that there is $L$ such that the functions $f_t$ are $L$-Lipschitz and $\Dom f_t\supset B(p,\eps\cdot L)$ for any $t\in[a,a+\eps)$.
Consider sequence  $a=t_0<t_1\dots<t_n\z=a+\eps$ and a piecewise constant family of functions on $B(p,\eps\cdot L)$ defined by $\hat f_t=f_{t_i}$ if $t_i\le t<t_{i+1}$.
By above discussion we have that there is a unique $\hat f_t$-gradient curve $\hat \alpha$ that starts at $p$ and defined on the interval $[a,a+\eps)$.

Note that the distance estimates (\ref{Distance estimate}) show that as the partition gets finer, the gradinet curves $\hat\alpha$ form a Cauchy sequence; denote its limit by $\alpha$.

Note that 
\begin{align*}
\dist_p\circ\hat\alpha(t+\eps)
&\le 
\dist_p\circ\hat\alpha(t)
-\eps\cdot \left[\tfrac{\hat f_t(p)-\hat f_t\circ\hat\alpha(t)}{|p-\alpha(t)|}-\tfrac\lambda2\cdot |p-\hat\alpha(t)|\right]
+o(\eps)
\\
&\le 
\dist_p\circ\hat\alpha(t)
-\eps\cdot \left[\tfrac{f_t(p)-f_t\circ\hat\alpha(t)-2\cdot s}{|p-\alpha(t)|}-\tfrac\lambda2\cdot |p-\hat\alpha(t)|\right]
+o(\eps)
\end{align*}
where 
\[s=\sup_{t,x} \{|f_t(x)-\hat f_t(x)|\}.\]
Since $s\to 0$ as $\hat\alpha\to \alpha$, we get that \ref{def:tdflow-plus} holds for $\alpha$;
that is, $\alpha$ is an $f_t$-gradient curve.
\qeds



{\small\sloppy

\printbibliography[heading=bibintoc]

}

\end{document}
