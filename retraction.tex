\documentclass[oneside,a4paper, 12pt]{article}
\usepackage{anysize}
\usepackage{retraction}
\usepackage{biber}

\hypersetup{pdftitle={Short retraction in CAT(1) space},
pdfauthor={Alexander Lytchak and Anton Petrunin}}

\begin{document}
\title{Short retraction in CAT(1) space}
\author{Alexander Lytchak and Anton Petrunin}
\maketitle

\begin{thm}{Theorem}\label{thm:retraction}
Let $\spc{U}$ be a complete length $\CAT(1)$
space.
\begin{enumerate}[(a)]
\item\label{thm:retraction:Phi} Suppose $K$ is a weakly convex closed subset in $\spc{U}$ and there is a point $p\in K$ such that $|p-x|\le \tfrac\pi2$ for any point $x\in K$.
Then there is a short retraction 
\[\Phi\:\spc{U}\to K.\]
\item\label{thm:retraction:Psi} Suppose there is there is a point $p\in \spc{U}$ such that $|p-x|\le \tfrac\pi2$ for any point $x\in \spc{U}$.
Then there is a short retraction 
\[\Psi\:\spc{U}\times \spc{U}\to \Delta,\]
 where $\Delta$ is the diagonal in $\spc{U}\times \spc{U}$;
 that is, $\Delta=\{\,(x,x)\in \spc{U}\times \spc{U}\,\}$.
\end{enumerate}

\end{thm}

The theorem makes possible to extend some results from $\CAT(0)$ spaces to small $\CAT(1)$ spaces.
It gives a more geometric version of Theorem A in \cite{yokota} which has nearly identical set of applications; see also \cite[(4.1)]{kendall}.  


\section{Part (a)}

The idea of the following construction is taken from the proof of Kirszbraun's theorem; see \cite[5.1]{akp-kirszbraun} or \cite[9.4.1]{akp}. %(\ref{thm:kirsz+}).

\begin{wrapfigure}{o}{55 mm}
\centering
\includegraphics{mppics/pic-1}
\caption{}\label{pic-1}
\end{wrapfigure}

\parit{Construction.}
Set $\mathring{\spc{U}}=\Cone \spc{U}$.
Denote by $\mathring{K}$ the subcone of $\mathring{\spc{U}}$ spanned by $K$.
The space $\spc{U}$ is the unit sphere in $\mathring{\spc{U}}$ with angle metric;
that is, $\spc{U}=\set{z\in \mathring{\spc{U}}}{|z|=1}$ and $|u-v|_{\spc{U}}=\measuredangle[o\,{}^u_v]$, where $o$ denotes the tip of the cone $\mathring{\spc{U}}$.

Note that $\mathring{\spc{U}}$ is a $\CAT(0)$ space and $\mathring{K}$ forms a convex closed subset of $\mathring{\spc{U}}$.
In particular, for any point $x$ there is unique point $\hat x\in \mathring{K}$
that minimize the distance to $x$;
that is, $|\hat x-x|=\dist_K(x)$.

Consider the ray $\alpha_o\:t\mapsto t\cdot p$ in  $\mathring{\spc{U}}$.
Note that given $s\in \mathring{\spc{U}}$
the geodesics $[s\ t\cdot p]$ converge as $t\to\infty$ to a ray, 
say $\alpha_s\:[0,\infty)\to \mathring{\spc{U}}$ that is parallel to $\alpha_o$.

Note that if $|x|=1$, then $|\hat x|\le 1$.
By assumption for any $k\in \mathring{K}$ the function $t\mapsto |\alpha_k(t)|$ is monotonically increasing.
Therefore there is unique value $t_x\ge 0$ such that
$|\alpha_{\hat x}(t_x)|=1$;
set $x'=\alpha_{\hat x}(t_x)$ and define the map $\Phi$ as $x\mapsto x'$.

Evidently $x\mapsto x'$ is a retraction of $\spc{U}$ to $K$;
that is,
$\Phi(x)\in K$ for any $x\in \spc{U}$
and 
$\Phi(k)=k$ for any $k\in K$.
It remains to show that $x\mapsto x'$ is short; the latter is done by straightforward calculations. 


\parit{Calculations.}
Let us describe this map intrinsically in $\spc{U}$, without passing to the cone.

Note that if $\dist_K(x)\ge\tfrac\pi2$, then $\hat x=o$ and therefore $x'=p$.

Otherwise, if $\dist_K(x)<\tfrac\pi2$, %\ref{lem:closest point}, 
there is unique point $x^*\in K$ that minimizes distance to $x$;
that is, $|x^*-x|_{\spc{U}}=\dist_K(x)$.
Note that $x^*=\hat x/|\hat x|$.
Let us define $\ell_x$, $\phi_x$ and $\psi_x$ using the following identities:
\begin{align*}
\ell_x&=|p-x^*|_{\spc{U}},
\\
\phi_x&=\tfrac\pi2-|x^*-x|_{\spc{U}},
\\
\sin\psi_x&=\sin\phi_x\cdot\sin\ell_x, 
\ \ 0\le \psi_x\le \tfrac\pi2.
\intertext{Note that}
x'&=\gamma_x(\psi_x),
\end{align*}
where $\gamma_x$ denoted the geodesic from $p$ to $x^*$ in $\spc{U}$.

Given two points $x$ and $y$ in the $\tfrac\pi2$-neigborhood of $K$, set
\begin{align*}
r&=|x-y|
&
r'&=|x'-y'|
\\
d&=|x^*-y^*|
&
\alpha&=\angk{p}{x^*}{y^*}
\end{align*}

Note that 
\[\cos r\le 
\cos\phi_x\cdot\cos\phi_y
-
\cos d\cdot\sin\phi_x\cdot\sin\phi_y.\eqlbl{eq:cos(r)}\]

Indeed, if $x,y\notin K$,
then 
$\mangle\hinge{x^*}{x}{y*}, 
\mangle\hinge{y^*}{y}{x*}
\ge 
\tfrac\pi2$
and
the inequality~\ref{eq:cos(r)} follows from the Arm lemma. % (\ref{lem:arm}).
If $x\in K$ and $y\notin K$, we get \ref{eq:cos(r)}, by angle comparison %(\ref{cat-hinge}) 
since $\mangle\hinge{y^*}{y}{x*}\ge \tfrac\pi2$.
The same way \ref{eq:cos(r)} is proved 
in case $x\notin K$ and $y\in K$.
Finally, if $x,y\in K$, $\phi_x=\phi_y=\tfrac\pi2$ and $r=d$;
that is, the inequality trivially holds.

Further note that
\[\cos\alpha
=
\frac{\cos d-\cos \ell_x\cdot\cos\ell_y}{\sin\ell_x\cdot\sin\ell_y}.\]
Applying angle-sidelength  monotonicity %(\ref{cor:monoton-cba}) 
we get
\[\begin{aligned}
\cos r'&\ge
\cos\psi_x\cdot\cos\psi_y
-
\cos \alpha \cdot\sin\psi_x\cdot\sin\psi_y=
\\
&=
\cos\psi_x\cdot\cos\psi_y
-(\cos d-\cos \ell_x\cdot\cos\ell_y)\cdot\sin\phi_x\cdot\sin\phi_y\ge
\\
&\ge \cos\psi_x\cdot\cos\psi_y
-\cos d\cdot\sin\phi_x\cdot\sin\phi_y
\end{aligned}
\eqlbl{eq:cos(r')}
\]


Note that 
$\psi_x\le \phi_x<\tfrac\pi2$
and
$\psi_y\le \phi_y<\tfrac\pi2$;
in particular,
\[
\cos\phi_x\cdot\cos\phi_y\le \cos\psi_x\cdot\cos\psi_y.
\]
Therefore \ref{eq:cos(r)} and \ref{eq:cos(r')} implies 
\[\cos r'\ge \cos r;\]
that is, the restriction of $x\mapsto x'$ to the $\tfrac\pi2$-neighborhood of $K$ is short.
Clearly the map $x\mapsto x'$ is continuous,
since the complement of the neighborhood is mapped to $p$,
we get that $\Phi$ is short; that is,
\[|x'-y'|\le |x-y| \eqlbl{eq:cos=<cos}\]
for any $x,y\in\spc{U}$.
\qeds

\section{Reduction (b) to (a)}

Recall that spherical join $\spc{U}\star\spc{V}$ of two metric spaces $\spc{U}$ and $\spc{V}$
is defined as the unit sphere equipped with the angle metric in the product of Euclidean cones $\Cone \spc{U}\times \Cone\spc{V}$.
It can be also defined as a metric space that admits an onto map $\iota\:\spc{U}\times\spc{V}\times[0,\tfrac\pi2]\to \spc{U}\star\spc{V}$ such that
\begin{align*}
|\iota(u_1,v_1,t_1)&-\iota(u_2,v_2,t_2)|_{\spc{U}\star\spc{V}}=
\\
&=\arccos[\sin t_1\cdot\sin t_2\cdot \cos|u_1-u_2|_{\spc{U}}+\cos t_1\cdot \cos t_2\cdot \cos|v_1-v_2|_{\spc{V}}];
\end{align*}
abusing the notation, we assume that $\cos x=-1$ if $x\ge \pi$.

From now on we will be interested in the space $\spc{U}\star\spc{U}$, where $\spc{U}$ is the space as in (\ref{thm:retraction:Psi}).
It is straightforward to see that 
\[|\iota(u_1,v_1,\tfrac\pi2)-\iota(u_2,v_2,\tfrac\pi2)|^2_{\spc{U}\star\spc{U}}\le |u_1-u_2|^2_{\spc{U}}+|v_1-v_2|_{\spc{U}}^2;\]
that is, the map $(u,v)\mapsto \iota(u,v,\tfrac\pi4)$ induces a short map 
\[\Theta\:{\tfrac1{\sqrt{2}}}\cdot(\spc{U}\times\spc{U})\to\spc{U}\star\spc{U}.\]
(The map $\Theta$ is also length preserving, but we will not need it.)

Note that the diagonal $\tfrac1{\sqrt{2}}\cdot\Delta$ is a convex set in $\tfrac1{\sqrt{2}}\cdot(\spc{U}\times\spc{U})$.
Moreover the restriction of $\Theta$ to $\tfrac1{\sqrt{2}}\cdot\Delta$ is distance preserving;
in particular the image $\Theta(\tfrac1{\sqrt{2}}\cdot\Delta)$ is a weakly convex set in $\spc{U}\star\spc{U}$.
Applying part (\ref{thm:retraction:Phi}) we get a short retraction $\spc{U}\star\spc{U}\to \Theta(\tfrac1{\sqrt{2}}\cdot\Delta)$.
Since $\Theta$ is short, it induces the needed short retraction $\Psi\:\spc{U}\times\spc{U}\to \Delta$.


{\small\sloppy

\printbibliography[heading=bibintoc]

}

\end{document}
